	\newpage
\section{Projektowanie}		%3
%Napisać z jakich narzędzi będziemy korzystać (kompilator, język programowania), git, biblioteki dodatkowe, itp.
%Opisać szczegółowe ustawienia kompilatora (jeśli są), powiązania z bibliotekami, itp.
%Narysować graf, UML, diagram klas, schemat działania algorytmu
%Jeśli zadanie zakłada przedstawienie jakiegoś narzędzia (np. git, AI) należy opisać sposób jego używania

\subsection{Implementacja listy}

Do zaimplementowania listy zostanie użyty Język \texttt{C++} z kompilatorem \texttt{g++}. Wersja standardu \texttt{C++}to \texttt{C++23}. Wersja ta została użyta, ze względu na zawartą w niej funkcję \texttt{std::print()}. Jako, że projekt ma być rozdzielony na dwa pliki, zostanie zastosowany \texttt{CMake} w celu automatyzacji procesu budowania. \texttt{CMake} pozwala na generowanie plików budujących dany projekt, zgodnie z określoną konfiguracją. Oszczędza to programiście, szczególnie przy większych projektach, manualne pisanie Makefileów. Edytorem będzie program Neovim. Jest to terminalowy edytor tekstu z możliwością poszerzenia funkcjonalności przy użyciu wszelkiego rodzaju pluginów. Wybrany został, dlatego że jest on już skonfigurowany na moim komputerze zgodnie z moimi preferencjami. 

\subsection{Git}

Dla ułatwienia pracy, zastosowany został front-end dla gita o nazwie lazygit. Jest to terminalowy program, którego główną zaletą jest łatwa nawigacja przy użyciu klawiatury.Ponadto, jest on lekki i szybki.

\subsection{Doxygen}

Konfiguracja dla Doxygena jest wygenerowana przy użyciu programu doxywizard, pozwalającego na graficzne zmienianie ustawień. Po wygenerowaniu konfiguracji, doxygen wywoływany jest komendą. 
