	\newpage
\section{Projektowanie}		%3
%Napisać z jakich narzędzi będziemy korzystać (kompilator, język programowania), git, biblioteki dodatkowe, itp.
%Opisać szczegółowe ustawienia kompilatora (jeśli są), powiązania z bibliotekami, itp.
%Narysować graf, UML, diagram klas, schemat działania algorytmu
%Jeśli zadanie zakłada przedstawienie jakiegoś narzędzia (np. git, AI) należy opisać sposób jego używania

\subsection{Implementacja listy}

Do zaimplementowania listy zostanie użyty Język C++ z kompilatorem g++. Jako, że projekt ma być rozdzielony na dwa pliki, zostanie zastosowany CMake w celu automatyzacji procesu budowania. Edytorem będzie program Neovim. 

\subsection{Git}

Dla ułatwienia pracy, zastosowany został front-end dla gita o nazwie lazygit. Jest to terminalowy program, którego główną zaletą jest łatwa nawigacja przy użyciu klawiatury.

\subsection{Doxygen}

Konfiguracja dla Doxygena jest wygenerowana przy użyciu programu doxywizard, pozwalającego na graficzne zmienianie ustawień. 
