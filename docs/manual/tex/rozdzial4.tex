	\newpage
\section{Implementacja}		%4
%Opisać implementacje algorytmu/programu. Pokazać ciekawe fragmenty kodu
%Opisać powstałe wyniki (algorytmu/nrzędzia)

\subsection{Ogólne informacje o implementacji klasy}

Lista jest zaimplementowana jako jeden plik \texttt{.hpp}. Nie jest podzielona na plik implementacji oraz nagłówek, ponieważ jest ona szablonem. Deklaracja klasy oraz prywatne elementy wyglądają następująco:

\begin{lstlisting}[caption=Deklaracja szblonu listy, label={lst:list_dec}, language=C++]
	template<typename T>
	class DoubleLinkedList {
	private:
	struct Node {
		Node* previous;
		Node* next;
		T contents;

		Node(Node* oPrevious, Node* oNext, T oContents): 
			previous { oPrevious },
			next { oNext },
			contents { oContents } {}

		~Node(void) {
			delete next;
		}
	};

	Node* head;
	Node* tail;

	Node* get_node_at(size_t index) {
		Node* currentNode { head };
		for(size_t i {}; i < index; i++) {
			currentNode = currentNode->next;
		}

		return currentNode;
	}

public:
	...
\end{lstlisting}

Jak widać w kodzie \ref{lst:list_dec}., Klasa jest wrapperem dla structa \texttt{Node}. Struct ten ma dwa wskaźniki - \texttt{next} dla elementu następnego i \texttt{previous} dla elementu poprzedniego. Jego konstruktor pozwala od razu ustawiać zawartość oraz te wskaźniki Metoda \texttt{get\_node\_at()} jest pomocniczą metodą pozwalającą uzyskać wskaźnik do \texttt{Node} o danym indeksie, podążając w przód od wskaźnika head, \texttt{index} razy.

Manipulacje strukturą listy odbywają się poprzez szereg metod publicznych. Wiele z nich posiada podobną strukturę. Za przykład jednej z nich można wziąć \texttt{prepend()}:

\begin{lstlisting}[caption = Kod \texttt{prepend()}, label={lst:prepend}, language=C++]
	void prepend(const T& item) {
		if(head == nullptr) {
			head = new Node { nullptr, nullptr, item };	
			tail = head;
			return;
		}

		head = new Node { nullptr, head, item };
		head->next->previous = head;
	}
\end{lstlisting}

Metoda z fragmentu nr. \ref{lst:prepend} ma na celu wstawienie elementu, którego wartość jest zawarta w parametrze \texttt{item} na początku listy. Na początku metody, sprawdzane jest czy lista jest pusta - wtedy \texttt{head == nullptr}. Jeżeli jest, trzeba stworzyć nowego \texttt{Node} na miejscu heada z zawartością będąca parametrem \texttt{item}. Jeżeli \texttt{head} już istnieje, to też tworzy się nowego \texttt{Nodea} na jego miejscu, lecz jako wskaźnik \texttt{next} ustawia się adres starego \texttt{heada}. Potem we wskaźniku \texttt{previous} starego \texttt{heada} ustawia się adres nowego heada. Ten zabieg efektywnie sprawił ze stary \texttt{head} jest drugi w kolejności listy.

\subsection{Ciekawe fragmenty kodu}

W metodzie \texttt{pop\_at()}, mającej na celu usuniecie \texttt{Nodea} o danym indeksie, wywoływany jest destruktor danego \texttt{Nodea} w taki sposób, aby - ze względu na jego rekursywny charakter - nie usuwać następnych elementów listy. Przy czym ciągłość listy musi być zachowana. Kod metody wygląda następująco:

\begin{lstlisting}[caption=Kod \texttt{pop\_at()}, label={lst:popat}, language=C++]

	void pop_at(size_t index) {
		Node* toPop { get_node_at(index) };
		if(toPop == head) {
			rpop();
			return;
		} else if(toPop == tail) {
			pop();
			return;
		}

		toPop->previous->next = toPop->next;
		toPop->next->previous = toPop->previous;
		toPop->next = nullptr;

		delete toPop;
	}

\end{lstlisting}

Jak widać na fragmencie nr. \ref{lst:popat}, takie zabiegi wymagają niezłej zabawy ze wskaźnikami. 
