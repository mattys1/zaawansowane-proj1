%╔════════════════════════════╗
%║	  Szablon dostosował	  ║
%║	mgr inż. Dawid Kotlarski  ║
%║		  06.10.2024		  ║
%╚════════════════════════════╝
\documentclass[12pt,twoside,a4paper,openany]{article}

    \input{preambula_pakiety.tex}
    \input{preambula_ustawienia.tex}

    %polecenia zdefiniowane w pakiecie strona_tytulowa.sty
    \title{...Algorytm listy dwukierunkowej \\z zastosowaniem GitHub...}		%...Wpisać nazwę projektu...
    \author{Mateusz Stanek}
    \authorI{}
    \authorII{}		%jeśli są dwie osoby w projekcie to zostawiamy:    \authorII{}
		
	\uczelnia{AKADEMIA NAUK STOSOWANYCH \\W NOWYM SĄCZU}
    \instytut{Wydział Nauk Inżynieryjnych}
    \kierunek{Katedra Informatyki}
    \praca{DOKUMENTACJA PROJEKTOWA}
    \przedmiot{ZAAWANSOWANE PROGRAMOWANIE}
    \prowadzacy{mgr inż. Dawid Kotlarski}
    \rok{2024}


%definicja składni mikrotik
\usepackage{fancyvrb}
\DefineVerbatimEnvironment{MT}{Verbatim}%
{commandchars=\+\[\],fontsize=\small,formatcom=\color{red},frame=lines,baselinestretch=1,} 
\let\mt\verb 
%zakonczenie definicji składni mikrotik

\usepackage{fancyhdr}    %biblioteka do nagłówka i stopki

			
\begin{document}
   
    \renewcommand{\figurename}{Rys.}    %musi byc pod \begin{document}, bo w~tym miejscu pakiet 'babel' narzuca swoje ustawienia
    \renewcommand{\tablename}{Tab.}     %j.w.
    \thispagestyle{empty}               %na tej stronie: brak numeru
    \stronatytulowa                     %strona tytułowa tworzona przez pakiet strona_tytulowa.tex
 
 \pagestyle{fancy}

    \newpage

    %formatowanie spisu treści i~nagłówków
    \renewcommand{\cftbeforesecskip}{8pt}
    \renewcommand{\cftsecafterpnum}{\vskip 8pt}
    \renewcommand{\cftparskip}{3pt}
    \renewcommand{\cfttoctitlefont}{\Large\bfseries\sffamily}
    \renewcommand{\cftsecfont}{\bfseries\sffamily}
    \renewcommand{\cftsubsecfont}{\sffamily}
    \renewcommand{\cftsubsubsecfont}{\sffamily}
    \renewcommand{\cftparafont}{\sffamily}
    %koniec formatowania spisu treści i nagłówków
     
    \tableofcontents    %spis treści
    \thispagestyle{fancy}
    \newpage

    
    \newpage

    
%%%%%%%%%%%%%%%%%%% treść główna dokumentu %%%%%%%%%%%%%%%%%%%%%%%%%

   	\newpage
\section{Ogólne określenie wymagań}		%1
%Określenie celu pracy, co chcemy uzyskać, jakie przewidujemy wyniki

\hspace{0.60cm}

Celem projektu jest napisanie programu implementującego listę dwukierunkową oraz kontrolowanie jego wersji za pomocą narzędzia git. Lista ma być zaimplementowana za pomocą klasy zawierającej się w innym pliku. Należy także wygenerować automatyczną dokumentację programu za pomocą narzędzia doxygen.  

Wynikiem projektu powinna być działająca klasa z implementacją listy, repozytorium git z jej kodem oraz dokumentacja w doxygenie.

   \newpage
\section{Analiza problemu}		%2
%Napisać gdzie używa się tego algorytmu
%Opisać sposób działania programu/algorytmu
%Napisać spsoób wykorzystania algorytmu po przez wykonanie przykładu (np. mnożenie macierzy - wykonać ręcznie przykład z mnożeniem macierzy pokazujący jak mnoży się macierz ręcznie)
%Jeśli zadanie zakłada przedstawienie jakiegoś narzędzia (np. git, AI) należy opisać narzędzie

\subsection{Lista}

List używa się przy okazjach, gdy potrzebny jest kontener potrafiący w szybki i prosty sposób modyfikować swoja wielkość i wewnętrzną strukturę, a szybkość dostępu do samych elementów nie jest aż tak ważna.

Lista jest zbiorem połączonych liniowo ze sobą elementów, gdzie w przypadku tej implementacji, elementy są połączone obustronnie t.zn., każdy element jest połączony z poprzednim i następnym. Dostęp do danego elementu uzyskuje się poprzez enumeracje po kolei wszystkich elementów w liście, aż nie dojdzie się do docelowego. Struktura taka pozwala na łatwe usuwanie i dodawanie elementów - wymaga to tylko zmienienia kilku wskaźników, a nie "przesuwania" całego kontenera. Graficzną reprezentację takiej listy widać na rysunku \ref{fig:list_struct}:

\begin{figure}[H]
	\centering
	\includegraphics[width=1\textwidth]{images/lista.drawio.png}
	\caption{\centering{Graficzna struktura listy dwukierunkowej}}
	\label{fig:list_struct}
\end{figure}

Strzałki między wskaźnikami na rysunku \ref{fig:list_struct}, symbolizują do jakich miejsc w pamięci wskazują wskaźniki.

Gdybyśmy chcieli dodać jakiś element do listy, proces wyglądałby jak na rys. \ref{fig:}:

\begin{figure}[H]
	\centering
	\includegraphics[width=1\textwidth]{images/listadodanie_pre.drawio.png}
	\caption{\centering{Element przed dodaniem do listy}}
	\label{fig:list_add_pre}
\end{figure}

Na rys. \ref{fig:list_add_pre} widać połączenia między elementami listy. \texttt{elem\_2} ma być dodany między \texttt{elem\_1} i \texttt{tail}. 

\begin{figure}[H]
	\centering
	\includegraphics[width=1\textwidth]{images/listadodanie_post.drawio.png}
	\caption{\centering{Element po dodaniu}}
	\label{fig:list_add_post}
\end{figure}

Jak widać z rys. \ref{fig:list_add_post}, wystarczy tylko zmienić adresy na jakie wskazują wskaźniki w danych pozycjach i element zostanie dodany.
\subsection{Git}
Kolejnym konceptem, którym zajmuje się projekt jest narzędzie git. Pozwala ono zarządzać poszczególnymi wersjami projektów. Głównym korzeniem gita jest system commitów, czyli zapisania zmian w pliku w stosunku do commita starszego. To, w połączeniu z jego innymi możliwościami pozwala na tworzenie długich i skomplikowanych osi czasu danych projektów. 

\subsection{Doxygen}
Doxygen jest narzędziem automatycznie generującym dokumentację programu z komentarzy w kodzie źródłowym. Potrafi on generować strony HTML, gdzie można dynamicznie nawigować się miedzy rożnymi częściami kodu oraz pliki \LaTeX, które można konwertować na różne, statyczne formaty.

   	\newpage
\section{Projektowanie}		%3
%Napisać z jakich narzędzi będziemy korzystać (kompilator, język programowania), git, biblioteki dodatkowe, itp.
%Opisać szczegółowe ustawienia kompilatora (jeśli są), powiązania z bibliotekami, itp.
%Narysować graf, UML, diagram klas, schemat działania algorytmu
%Jeśli zadanie zakłada przedstawienie jakiegoś narzędzia (np. git, AI) należy opisać sposób jego używania

\subsection{Implementacja listy}

Do zaimplementowania listy zostanie użyty Język C++ z kompilatorem g++. Wersja standardu C++ to C++23. Wersja ta została użyta, ze względu na zawartą w niej funkcję std::print(). Jako, że projekt ma być rozdzielony na dwa pliki, zostanie zastosowany CMake w celu automatyzacji procesu budowania. CMake pozwala na generowanie plików budujących dany projekt, zgodnia z określoną konfiguracją. Oszczędza to programiście, szczegolnie przy wiekszych projektach, manualne pisanie Makefileów. Edytorem będzie program Neovim. Jest to terminalowy edytor tekstu z możliowoscia poszerzenia funkcjonalności przy uzyciu wszelkiego rodzaju pluginow. Wybrany zostal, dlatego że jest on juz skonfigurowany na moim komputerze zgodnie z moimi preferencjami. 

\subsection{Git}

Dla ułatwienia pracy, zastosowany został front-end dla gita o nazwie lazygit. Jest to terminalowy program, którego główną zaletą jest łatwa nawigacja przy użyciu klawiatury.Ponadto, jest on lekki i szybki.

\subsection{Doxygen}

Konfiguracja dla Doxygena jest wygenerowana przy użyciu programu doxywizard, pozwalającego na graficzne zmienianie ustawień. Po wygenerowaniu konfiguracji, doxygen wywolywany jest komendą. 

   	\newpage
\section{Implementacja}		%4
%Opisać implementacje algorytmu/programu. Pokazać ciekawe fragmenty kodu
%Opisać powstałe wyniki (algorytmu/nrzędzia)

Lista jest zaimplementowana jako jeden plik .hpp. Nie jest podzielona na plik implementacji oraz nagłówek, ponieważ jest ona szablonem. Deklaracja klasy wygląda następująco

   	\newpage
\section{Wnioski}	%5
%Npisać wnioski końcowe z przeprowadzonego projektu, 

\begin{itemize}
	\item Przy rebaseowaniu, należy zwrócić uwagę, jakie pliki zostaną zmienione. 
	\item Stashe w git są zbawieniem.
\end{itemize}


   
       
%%%%%%%%%%%%%%%%%%% koniec treść główna dokumentu %%%%%%%%%%%%%%%%%%%%%
	\newpage
    \addcontentsline{toc}{section}{Literatura}  
	\printbibliography

    \newpage
    \hypersetup{linkcolor=black}
    \renewcommand{\cftparskip}{3pt}
    \clearpage
    \renewcommand{\cftloftitlefont}{\Large\bfseries\sffamily}
    \listoffigures
    \addcontentsline{toc}{section}{Spis rysunków}
	\thispagestyle{fancy}
	
    \newpage
    \renewcommand{\cftlottitlefont}{\Large\bfseries\sffamily}
    \def\listtablename{Spis tabel}
    \addcontentsline{toc}{section}{Spis tabel}\listoftables 
	\thispagestyle{fancy}
	
	\newpage
	\renewcommand{\cftlottitlefont}{\Large\bfseries\sffamily}
	\renewcommand\lstlistlistingname{Spis listingów}
	\addcontentsline{toc}{section}{Spis listingów}\lstlistoflistings 
	\thispagestyle{fancy}
	


    %lista rzeczy do zrobienia: wypisuje na koñcu dokumentu, patrz: pakiet todo.sty
    \todos
    %koniec listy rzeczy do zrobienia
\end{document}
